\begin{comment}
--Summary of main part of text
- Deduction made of basis of main body
- Personal opinion of what has been discussed
- Comment about the future
- implications for future work
- important facts and figures not mentioned in body of work
- summary of main points
- concluding Statements
- recommendations
- predictions
- solutions
\end{comment}

\chapter{Conclusion and Outlook}
The aim of this thesis has been to provide proof of concept for discovery algorithms using only fundamental building blocks of the so-called \textit{Web of Things}, with particular emphasis on the industrial application thereof.

A brief overview of relevant Semantic Web technology has been provided.

The central focus of this thesis was the challenge of leveraging Semantic Web technologies, RDF and SPARQL, to create semi-automated discovery algorithms, the feasibility thereof was demonstrated in \ref{chap:experiment}.

In the past year, Siemens along with other researchers have continued to develop W3C recommendations especially Semantic Web recommendations, which will form the backbone of the Web of Things.

